\documentclass[11pt,a4paper]{article}

% French
\usepackage[utf8x]{inputenc}
\usepackage[frenchb]{babel}
\usepackage[T1]{fontenc}
\usepackage{lmodern}
\usepackage{ifthen}
%\usepackage{parskip}
%\setlength{parskip}{0cm}
\setlength{\parindent}{0cm}

% Color
% cfr http://en.wikibooks.org/wiki/LaTeX/Colors
\usepackage{color}
\usepackage[usenames,dvipsnames,svgnames,table]{xcolor}
\definecolor{dkgreen}{rgb}{0.25,0.7,0.35}
\definecolor{dkred}{rgb}{0.7,0,0}

% Listing
\usepackage{listings}
\lstset{
  numbers=left,
  numberstyle=\tiny\color{gray},
  basicstyle=\rm\small\ttfamily,
  keywordstyle=\bfseries\color{dkred},
  frame=single,
  commentstyle=\color{gray}=small,
  stringstyle=\color{dkgreen},
  %backgroundcolor=\color{gray!10},
  %tabsize=2,
  rulecolor=\color{black!30},
  %title=\lstname,
  breaklines=true,
  framextopmargin=2pt,
  framexbottommargin=2pt,
  extendedchars=true,
  inputencoding=utf8x
}
\newcommand{\oz}{\textsc{Oz}}
\newcommand{\java}{\textsc{Java}}
\newcommand{\clang}{\textsc{C}}
\newcommand{\keyword}{mot clef}

% Math symbols
\usepackage{amsmath}
\usepackage{amssymb}
\usepackage{amsthm}
\DeclareMathOperator*{\argmin}{arg\,min}
\DeclareMathOperator*{\argmax}{arg\,max}

% Sets
\newcommand{\Z}{\mathbb{Z}}
\newcommand{\R}{\mathbb{R}}
\newcommand{\Rn}{\R^n}
\newcommand{\Rnn}{\R^{n \times n}}
\newcommand{\C}{\mathbb{C}}
\newcommand{\K}{\mathbb{K}}
\newcommand{\Kn}{\K^n}
\newcommand{\Knn}{\K^{n \times n}}

\usepackage{fullpage}

% Questions
\usepackage{xparse}
\NewDocumentEnvironment{Question}{mm}
{\subsection*{LINGI1131 -- Final examination\\Prof. Peter Van Roy\\\date}
  \begin{flushright}
    \begin{tabular}{|p{4.3cm}|p{7.4cm}|}
      \hline
      Last name & \\
      \hline
      First name & \\
      \hline
      Student number (NOMA) & \\
      \hline
    \end{tabular}
  \end{flushright}
\subsubsection*{Question #1 : #2 (5 pts)}}
{\thispagestyle{empty}\clearpage}

\NewDocumentEnvironment{Q1}{m}
{\begin{Question}{1}{#1}
{\it For this question, your points will be the maximum of the points of your answer and your points on the midterm.}

}
{\end{Question}}

\NewDocumentEnvironment{Q2}{m}
{\begin{Question}{2}{#1}}
{\end{Question}}

\NewDocumentEnvironment{Q3}{m}
{\begin{Question}{3}{#1}}
{\end{Question}}

\begin{document}

\def\date{Jun. 20, 2014}

\begin{Q1}{An oscillator with delay gates}
  For this question, do the following:
  \begin{itemize}
    \item Implement a Not gate and a Delay gate using the simulation technique used in the course.
    \item Give the code for the circuit shown in the figure.
    \item What is the output stream of this circuit.
    \item Prove rigorously your answer to the previous point.
      For the proof,
      represent the streams symbolically, for example
      A = $a_0$|$a_1$|$\cdots$,
      and analogously for B and C.
      Write the equations for the circuit in the figure.
      Solve the equations to get the values of all output stream
      elements.
    \item Is the output stream periodic?
      That is, does there exist a $p$ such that
      $\forall i \geq 0: a_{i+p}=a_i$.
      If so, what is the period of the circuit (the smallest value of $p$)?
  \end{itemize}
  Make sure to use correct syntax in all your program code (syntax errors will be graded severely!),
  and indent your programs to improve their readability.
  Explain clearly and concisely how your program work.
  You will be graded not just on the correctness of your programs
  but also on their conciseness, their presentation, and on your explanations.
\end{Q1}

\begin{Q2}{Random collisions}
  \begin{itemize}
    \item For the first part of this question,
      implement an $n \times n$ square array of agents.
      Each agent has four neighbors,
      which are the adjacent agents left, right, above, and below.
      Agents in the rightmost column have as right neighbors the agents
      in the leftmost columnt and vice versa.
      Agents in the top row have as top neighbors the agents in the bottom row and vice versa.
      This means that the $n \times n$ array is in fact a \emph{torus}, i.e., a donut shape.
      Give the code to define the torus.
    \item For the second part of this question,
      implement a ``running message''.
      When any agent receives the message \lstinline|ball(N)| from a neighbor,
      it will pick one of the other neighbors and forward the message
      \lstinline|ball(N)| to that neighbor.
      The input argument \lstinline|N| is passed to the output.
      Give the code to define the agent.
      You can assume the function \lstinline|{Rand N}| exists
      that takes a positive integer \lstinline|N| and returns
      a pseudorandom number from $0$ to $n-1$.
    \item For the third part of this question,
      give the code to create a system with a 10 by 10 torus
      and 100 ball messages numbered \lstinline|ball(0)|
      to \lstinline|ball(99)|.
      The ball messages are initially sent to all different
      agents.
      This system will have 100 balls traveling over it randomly.
    \item Now we get to the interesting part.
      For the fourth part of this question, extend the agents
      so their behavior depends on the previous ball number received.
      If this number is \lstinline|M| and the current ball
      received is \lstinline|M+1 mod 100|, then the agent swallows
      the current ball (it does not forward it).
      Note that the extended agent has an internal state.
      Give the code of the extended agent.
      Give also the state diagram of the extended agent.
    \item What is the final configuration of the whole system,
      after it has swallowed all the balls it can swallow?
  \end{itemize}
\end{Q2}

\begin{Q3}{Locks and monitors}
  For this question you will explore some different versions of the lock concept.
  \begin{itemize}
    \item Define the concept of a simple lock and give its implementation with dataflow variables and cells.
      Explain how your implementation uses the idea of token passing.
    \item Define the concept of a reentrant lock and give its implementation.
      Explain how your implementation extends the simple lock implementation.
      Give a scenario, including a code fragment, that explains why a reentrant lock is needed in practical programming systems and
      why a simple lock is insufficient.
    \item Define the concept of a monitor and give its programming interface and semantics.
      Explain how a monitor is an extension of a lock.
      Give a scenario of a program that cannot be implemented with reentrant locks but that needs a monitor.
    \item Explain how to use monitors in practical programs.
      Give a scenario, including a code fragment, to show how to use monitors.
      Explain why your implementation is correct for all thread interleavings.
      In particular, give some concrete examples of interleavings and explain how the implementation works in those cases
      (for example, in the course I presented a particularly interesting interleaving).
      Give the programming pattern that is used in your scenario.
  \end{itemize}
\end{Q3}

\end{document}
